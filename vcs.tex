% Init .tex document for a collections of questions of the VCS exam

\documentclass[12pt]{article}
\usepackage{amsmath}
\usepackage{amssymb}
\usepackage{amsfonts}
\usepackage{graphicx}
\usepackage{hyperref}
\usepackage{listings}
\usepackage{color}
\usepackage{float}
\usepackage{subcaption}
\usepackage{enumitem}


\title{VCS Exam Questions}
\author{Marco Bernardi}
\date{\today}

\begin{document}
\maketitle

\section{Introduction}
This document contains a collection of questions that can be used to prepare for the VCS exam.
\section{Machine Learning}
\begin {enumerate}
    \item \textbf{Describe the main components of a supervised machine learning model (e.g. linear regression), 
    i.e. the hypothesis function, the cost function and how the learning works.}


\end{enumerate}

\section{Filters}
\begin{enumerate}
    % Write in bold the question 
    \item \textbf{Please provide a clear definition for a Gaussian filter (i.e. the kernel mask) and introduce another filter that can give you a similar output.
    You should also describe the main differences between the two.}

    \item \textbf{You have this kernel:     16 * [1, 2, 1; 2, 4, 2; 1, 2, 1]
    Define what is it and another kernel with which you can achieve a similar result.}

    \item \textbf{What’s the purpose of image filtering?  You should give a clear definition of filtering and also define the mathematical tools used in order to define any filter.
    Finally, give anexample of a smoothing filter (and provide also its definition).}

    \item \textbf{Please provide i) a clear definition of the convolutional operator, ii) its main propertiesand iii) 
    report the kernel matrix of a filter that will produce as an output an image that is shifted by 2 pixels in the left direction.}

    Convolutional operator is a mathematical operation that takes two functions f and g and produces a third function that represents how the shape of one is modified by the other. 
    The convolution of f and g is written f * g. The convolution operator is often seen in signal processing, where it models the effect of a linear time-invariant system on a signal.
    Convolutional operator follows the classical mathematical properties of associativity, commutativity, and distributivity, homogeneity, shift invariance, and separability.
    In image processing, the convolution operator is used to apply filters to images. 
    The kernel matrix of a filter that will produce as an output an image that is shifted by 2 pixels in the left direction is:
    \[
    \begin{bmatrix}
    0 & 0 & 0 \\
    0 & 0 & 2 \\
    0 & 0 & 0
    \end{bmatrix}
    \]

    \item \textbf{What is this filter [1 0 -1; 2 0 -2; 1 0 -1]?  Please “decompose” it in its two 1D components and highlight their respective roles.}
    
    The following filter is a Sobel filter applied to vertical edges. The filter can be decomposed into two 1D components:
    \[
    \begin{bmatrix}
    1 \\
    2 \\
    1
    \end{bmatrix}
    \begin{bmatrix}
    1 & 0 & -1
    \end{bmatrix}
    \]
    The first component is a weighted average and scaling filter that emphasizes the vertical edges in the image.
    It's important to use a smoothing filter to reduce the noise in the image before applying the Sobel filter, because derivatives are sensitive to noise.
    The second component is a derivative filter that computes the gradient of the image in the vertical direction. 
    The combination of the two filters highlights the vertical edges in the image.

    Sobel filter can be used also for horizontal edges by swapping the two components of the filter.
\end{enumerate}

\section{Edge}

\begin{enumerate}
    % Write in bold the question 
    \item \textbf{Describe an algorithm for edge detection (and define the kernel used). Report eventual similarities between this procedure and human vision.}
    
    \item \textbf{Please provide an example of edge detector (i.e. the kernel matrix) and illustrate whatare the main analogies with biological vision (if any), 
    and the applications in which edgedetection could be a fundamental component.}

    \item \textbf{Describe in detail the SIFT algorithm; you should present both detection and description stages, 
    and illustrate the classical SIFT matching approach (aka Lowe’s 2NN ratio test).}

    \item \textbf{Describe all the major components of the SIFT algorithm and provide some examples ofapplications in which SIFT features are the key ingredient.}

    Ciao

\end{enumerate}

\section{BOVW}
\begin{enumerate}
    % Write in bold the question 
    \item \textbf{Describe in detail a bag-of-visual-words model by highlighting all the main steps of computation and the major “design choices” /parameters.
    Then you should briefly introducespatial pyramids and motivate their advantages with respect to a standard pipeline.}

    \item \textbf{Describe in details a bag-of-visual-words pipeline. What are its main stages and what are their roles?}

    Ciao
    \item \textbf{Describe the Bag of Word pipeline. Propose a technique to address feature locality.}
    
    \item \textbf{Describe in details a bag-of-visual-words pipeline highlighting all the main steps of com-putation. 
    Finally, you should briefly introduce how it can be extended in order to betterrepresent the spatial elements depicted in a given image.}

\end{enumerate}

\section{CNN}
\begin{enumerate}
    % Write in bold the question 
    \item \textbf{Describe in detail a CNN architecture, introducing the most important components (layers, activation functions, etc.)  
    and pros/cons with respect to a traditional neural network.}

    \item \textbf{Describe the main components of a CNN (conv, pooling and FC layers). Report a typical architecture (take example from AlexNet).}
    
    \item \textbf{What’s a perceptron? Describe in details its main components and how a multi-layerperceptron can add more representation power w.r.t. a simple perceptron.}   

\end{enumerate}
\end{document}